
\documentclass{article}

\usepackage[table]{xcolor}         % colors
\usepackage[utf8]{inputenc} % allow utf-8 input
\usepackage[T1]{fontenc}    % use 8-bit T1 fonts
\usepackage{hyperref}       % hyperlinks
\usepackage{url}            % simple URL typesetting
\usepackage{subcaption, booktabs}       % professional-quality tables
\usepackage{amsfonts}       % blackboard math symbols
\usepackage{nicefrac}       % compact symbols for 1/2, etc.
\usepackage{microtype}      % microtypography
\usepackage{amssymb, amsmath, amsthm, mathtools, mathrsfs} 
\usepackage{bbm}
\usepackage{todonotes}
\usepackage{multirow}
\usepackage{enumerate}
\usepackage{caption}
\usepackage{lipsum}
\usepackage{pdflscape}
\usepackage{booktabs}
\usepackage{multirow}
\usepackage{geometry}
\usepackage{titlesec}
\usepackage{optidef}

\usepackage{colortbl}
\usepackage{array}
\usepackage{bold-extra}

\usepackage{tikz}
\usetikzlibrary{matrix, positioning}

% Setup the caption for Tables and Figures
\captionsetup{labelfont=bf}


%%%%%%%%%%%%%%%%%%%%%%%%%%%%%%%%%%%%%%%%%%%%%%%%%%%%%%%%%%%
%%%%%%%%%%%%%%%%%%%%%%%%%%%%%%%%%%%%%%%%%%%%%%%%%%%%%%%%%%%
% My most commonly used math symbols 
%%%%%%%%%%%%%%%%%%%%%%%%%%%%%%%%%%%%%%%%%%%%%%%%%%%%%%%%%%%
%%%%%%%%%%%%%%%%%%%%%%%%%%%%%%%%%%%%%%%%%%%%%%%%%%%%%%%%%%%

\newcommand{\R}{\mathbb{R}} %real numbers 
\newcommand{\N}{\mathbb{N}} %natural numbers  
\newcommand{\Z}{\mathbb{Z}} %positive integers 
\newcommand{\Field}{\mathbb{F}} %field of numbers 
\newcommand{\im}{\mathbbm{i}} %imaginary number

% Probability symbols
\newcommand{\Prob}{\mathbb{P}} %Probability
\newcommand{\E}{\mathbb{E}} %Expectation
\newcommand{\Var}{\ensuremath{\mathrm{Var}}} %Variance
\newcommand{\Std}{\ensuremath{\mathrm{Std}}} %Standard deviation 
\newcommand{\Cov}{\ensuremath{\mathrm{Cov}}} %Covariance
\newcommand{\Corr}{\ensuremath{\mathrm{Corr}}} %Correlation
\newcommand{\inner}[2]{\ensuremath{\left\langle #1,#2 \right\rangle}} %Inner product / functional notation
\newcommand{\innersmall}[2]{\ensuremath{\langle #1,#2 \rangle}} %Inner product / functional notation
\newcommand{\norm}[1]{\vert\vert#1\vert\vert} %Norm function
\newcommand{\abs}[1]{|#1|} %Absolute value function
\newcommand{\normBIG}[1]{\left\vert\left\vert#1\right\vert\right\vert} %Big norm function
\newcommand{\absBIG}[1]{\left|#1\right|} %Big absolute value function
\newcommand{\diag}{\ensuremath{\mathrm{diag}}} % matrix diag operator  
\newcommand{\ind}{\ensuremath\mathbbm{1}} % indicator function 
\newcommand{\indsmall}[1]{\ensuremath\mathbbm{1}_{\{#1\}}} % small indicator function 
\newcommand{\weakcvgto}{\ensuremath\rightsquigarrow} % weak convergence arrow 
\newcommand{\BigOhPee}{\ensuremath O_\mathrm{P}} % Big O in probability 
\newcommand{\SmallOhPee}{\ensuremath o_\mathrm{P}} % Small o in probability 
\newcommand{\BigOh}{O} % Big O notation 
\newcommand{\SmallOh}{o} % Small O notation 
\newcommand{\normaldist}{\mathcal{N}}

% General functions 
\newcommand{\Erf}{\ensuremath\mathrm{erf}} % error function 
\newcommand{\Efrc}{\ensuremath\mathsf{efrc}} % complementary error function = 1 - erf 

% Functional analysis 
\newcommand{\Ran}{\ensuremath\mathrm{ran}\,} % range 
\newcommand{\Ker}{\ensuremath\mathrm{ker}\,} % kernel 
\newcommand{\ident}{\ensuremath\mathrm{id}} % identity operator 

% Analysis
\newcommand{\vol}{\ensuremath\mathrm{vol}} % volume 

% Calculus 
\newcommand{\diff}[1]{\ensuremath\,\mathrm{d}#1}
\newcommand{\Diff}{\ensuremath\mathrm{D}}

% Optimization
\DeclareMathOperator*{\argmax}{arg\,max}
\DeclareMathOperator*{\argmin}{arg\,min}
\DeclareMathOperator*{\argsup}{arg\,sup}
\DeclareMathOperator*{\arginf}{arg\,inf}

% Change the real and imaginary parts operators
\renewcommand{\Re}{\operatorname{Re}}
\renewcommand{\Im}{\operatorname{Im}}


%%%%%%%%%%%%%%%%%%%%%%%%%%%%%%%%%%%%%%%%%%%%%%%%%%%%%%%%%%%
%%%%%%%%%%%%%%%%%%%%%%%%%%%%%%%%%%%%%%%%%%%%%%%%%%%%%%%%%%%
% Symbols specific to this paper 
%%%%%%%%%%%%%%%%%%%%%%%%%%%%%%%%%%%%%%%%%%%%%%%%%%%%%%%%%%%
%%%%%%%%%%%%%%%%%%%%%%%%%%%%%%%%%%%%%%%%%%%%%%%%%%%%%%%%%%%

\newcommand{\VaR}{\ensuremath \mathrm{VaR}}

\renewcommand{\vec}{\ensuremath \operatorname{vec}}

\newcommand{\onevec}{\ensuremath \mathbf{1}}
\newcommand{\meanvec}{\ensuremath \boldsymbol{\mu}}
\newcommand{\covmat}{\ensuremath \boldsymbol{\Sigma}}

\newcommand{\utility}{\ensuremath U}
\newcommand{\numassets}{\ensuremath n}
\newcommand{\returns}{\ensuremath \mathbf{Y}}
\newcommand{\riskfreerate}{\ensuremath R^f}
\newcommand{\portfolio}{\ensuremath \boldsymbol{\pi}}
\newcommand{\portfoliospace}{\ensuremath \mathcal{P}}
\newcommand{\wealth}{\ensuremath W}
\newcommand{\riskloss}{\ensuremath L}
\newcommand{\riskmeasure}{\ensuremath \mathbf{v}}
\newcommand{\riskmeasurespace}{\ensuremath \mathrm{D}}
\newcommand{\riskmeasureconstraints}{\ensuremath \mathcal{V}}
\newcommand{\params}{\ensuremath \boldsymbol{\theta}}

\newcommand{\xvec}{\ensuremath \mathbf{x}}
\newcommand{\sxvec}{\ensuremath \mathrm{x}}

\newcommand{\Xmat}{\ensuremath \mathbf{X}}

\newcommand{\zvec}{\ensuremath \mathbf{z}}

\newcommand{\spread}{\ensuremath \text{spr}}

\newcommand{\sign}{\ensuremath \operatorname{sign}}

\newcommand{\MSE}{\ensuremath \text{MSE}}

\newcommand{\C}{\mathbb{C}} %complex numbers 

\SetLipsumParListSurrounders{\colorlet{oldcolor}{.}\color{orange}}{\color{oldcolor}}

\newcolumntype{P}[1]{>{\centering\arraybackslash}p{#1}}

% Unnumbered theorem styles 
\theoremstyle{plain}
\newtheorem*{empiricalimp*}{Empirical Hypothesis}
\newtheorem*{thm*}{Theorem}
\newtheorem*{exercise*}{Exercise} 
\newtheorem*{example*}{Example} 
\newtheorem*{discussion*}{Discussion} 
\newtheorem*{claim*}{Claim}
\newtheorem*{lem*}{Lemma}
\newtheorem*{prop*}{Proposition}
\newtheorem*{cor*}{Corollary}
%\newtheorem*{KL}{Klein's Lemma}
\newtheorem*{defn*}{Definition}
\newtheorem{conjecture}{Conjecture}[section]
\theoremstyle{remark}
\newtheorem*{rem*}{Remark}
\newtheorem*{note*}{Note}
\newtheorem{case*}{Case}



%% Results path
%\makeatletter
%\def\input@path{{../data/simulations/tex/}{../data/realworld/tex/}}
%%or: \def\input@path{{/path/to/folder/}{/path/to/other/folder/}}
%\makeatother
%
%\graphicspath{{../data/realworld/fig/}}


\title{\textcolor{red}{Dynamic Semiparametric Portfolio Choice with Elicitable Risk Constraints}}

\date{\textcolor{red}{Some date}}


% The \author macro works with any number of authors. There are two commands
% used to separate the names and addresses of multiple authors: \And and \AND.
%
% Using \And between authors leaves it to LaTeX to determine where to break the
% lines. Using \AND forces a line break at that point. So, if LaTeX puts 3 of 4
% authors names on the first line, and the last on the second line, try using
% \AND instead of \And before the third author name.


\author{%
  Raymond C.\ W.\ Leung\thanks{Hi} \\
  \texttt{raymond.chi.wai.leung@gmail.com} 
}

%%%%%%%%%%%%%%%%%%%%%%%%%%%%%%%%%%%%%%%%%%%%%%%%%%%%%%%%%%%%%%%%%%%%%%%%%%
%% Paper variables
%%%%%%%%%%%%%%%%%%%%%%%%%%%%%%%%%%%%%%%%%%%%%%%%%%%%%%%%%%%%%%%%%%%%%%%%%%
\newcommand{\DataStart}{\textcolor{red}{Data Start}}
\newcommand{\DataEnd}{\textcolor{red}{Data End}}
%%%%%%%%%%%%%%%%%%%%%%%%%%%%%%%%%%%%%%%%%%%%%%%%%%%%%%%%%%%%%%%%%%%%%%%%%%


\begin{document}


\maketitle


%%%%%%%%%%%%%%%%%%%%%%%%%%%%%%%%%%%%%%%%%%%%%%%%%%%%%%%%%%%%%%%%%%%%%%%%%%
\begin{abstract}
	\lipsum[77]
\end{abstract}
%%%%%%%%%%%%%%%%%%%%%%%%%%%%%%%%%%%%%%%%%%%%%%%%%%%%%%%%%%%%%%%%%%%%%%%%%%

%%%%%%%%%%%%%%%%%%%%%%%%%%%%%%%%%%%%%%%%%%%%%%%%%%%%%%%%%%%%%%%%%%%%%%%%%%
\section{Setup sketch}
%%%%%%%%%%%%%%%%%%%%%%%%%%%%%%%%%%%%%%%%%%%%%%%%%%%%%%%%%%%%%%%%%%%%%%%%%%
$\riskmeasurespace \subseteq \R^p$ and $\riskmeasureconstraints \subseteq \riskmeasurespace$

Consider the myopic problem
% \begin{align}
% 	% \begin{gathered}
% 	\max_{\portfolio \in \portfoliospace, \riskmeasure \in \riskmeasureconstraints}\; \E_{t - 1}\left[ \utility(\wealth_t^{\portfolio}) \right] equation \\
% 	\text{subject to:}                                                                                                                                   \\
% 	\wealth_t^{\portfolio} = w_0 ( \riskfreerate_t + \portfolio^\top ( \returns_t - \riskfreerate_t \ind_\numassets )  )                                 \\
% 	\riskmeasure \in \argmin_{\riskmeasure' \in \riskmeasurespace}\; \E_{t - 1} \left[ \riskloss( \riskmeasure', \wealth_t^{\portfolio}) \right]
% 	% \end{gathered}
% \end{align}


\begin{maxi!}|l|
{\portfolio \in \portfoliospace, \riskmeasure \in \riskmeasureconstraints} {\E_{t - 1}\left[ \utility(\wealth_t^{\portfolio}) \right] \label{eq:OrangeObjFunc} }{\label{eq:OrangeOptimProb}}{}
\addConstraint{\wealth_t^{\portfolio} = w_0 ( \riskfreerate_t + \portfolio^\top ( \returns_t - \riskfreerate_t \onevec_\numassets )) }{ \label{eq:OrangeWealth}}
\addConstraint{\riskmeasure \in \argmin_{\riskmeasure'}\; \E_{t - 1} \left[ \riskloss( \riskmeasure', \wealth_t^{\portfolio}) \right]}{ \label{eq:OrangeRiskIC} }
\end{maxi!}

%%%%%%%%%%%%%%%%%%%%%%%%%%%%%%%%%%%%%%%%%%%%%%%%%%%%%%%%%%%%%%%%%%%%%%%%%%
\section{Illustration}
%%%%%%%%%%%%%%%%%%%%%%%%%%%%%%%%%%%%%%%%%%%%%%%%%%%%%%%%%%%%%%%%%%%%%%%%%%
\todo{Find reference}
Suppose we have asset returns $\returns \sim \normaldist(\meanvec, \covmat)$. Suppose we have CARA utility $\utility(w) = - e^{-\gamma w}$ for some risk aversion parameter $\gamma > 0$. By conventional arguments,
\begin{equation}
	\E[\utility(W)] = -\exp\left\{ -\gamma w_0 \left[ \riskfreerate + \portfolio^\top (\meanvec - \riskfreerate \onevec_\numassets) - \frac{\gamma w_0}{2} \portfolio^\top \covmat \portfolio \right] \right\}
	\label{eq:CARAUtility}
\end{equation}

Fixing a level $\alpha \in (0,1)$, it is well known the the score function $S(x,y) = (\ind\{x \ge y\} - \alpha)(x - y)$ will lead to the $\alpha$-quantile. Hence, \eqref{eq:OrangeRiskIC} solves explicitly to
\begin{equation}
	\VaR_\alpha(\portfolio)
	= \argmin_{v'} \E[\riskloss(v', \wealth_t^{\portfolio})]
	= w_0\left(\riskfreerate + \portfolio^\top (\meanvec - \riskfreerate \onevec_\numassets ) + \sqrt{2} \Erf^{-1}(2\alpha - 1)  \sqrt{  \portfolio^{\top} \covmat \portfolio }   \right)
	\label{eq:CARAVaR}
\end{equation}
Suppose we have the constraint set $\riskmeasureconstraints = \{ v' \in \R : v' \le w_0 \bar{v} \}$ for some constant $\bar{v} \le 0$. Putting everything together, we reduce the optimization problem to

\begin{maxi}
	{\portfolio \in \R^\numassets}{ \riskfreerate + \portfolio^\top (\meanvec - \riskfreerate \onevec_\numassets) - \frac{\gamma w_0}{2} \portfolio^\top \covmat \portfolio }{}{}
	\addConstraint{ \riskfreerate + \portfolio^\top (\meanvec - \riskfreerate \onevec\numassets ) + \sqrt{2} \Erf^{-1}(2\alpha - 1)  \sqrt{  \portfolio^{\top} \covmat \portfolio } \le \bar{v} }
\end{maxi}

The constraint $v \in \riskmeasureconstraints$ behaves like an \emph{individual rationality} constraint in the principal-agent literature.
\footnote{
	Strictly speaking, we should formulate the individual rationality constraint in terms of the agent's utility function. In our current setup, it would take the form $\E[ \riskloss(v', \wealth^{\portfolio}) ] \ge \underbar{\riskloss}$ for some constant $\underbar{\riskloss}$.
}


\todo{Find reference for risk preference difference in principal-agent lit}
Our highly stylized principal-agent problem here illustrates the tension between the principal (``portfolio manager'') and the agent (``risk manager''). Under this stylized illustration, the principal is a mean-variance optimizer while the agent is a mean-standard deviation optimizer. Up to the VaR constraint of $\bar{v}$, the principal and the agent have the same first moment term $\riskfreerate + \pi^\top( \meanvec - \riskfreerate \onevec_n)$ in their preferences. Both the principal and the agent agree to choose portfolios $\portfolio$ that achieves a high mean wealth. However, the principal and the agent have different risk aversions. Let's assume no wealth effects for simplicity, so say $w_0 = 1$. Note that $\Erf^{-1} \le 0$ on $(-1, 0]$. Thus for conventional values of $\alpha$ (e.g.\ $0.05$), it is ensured that $\Erf^{-1}(2 \alpha - 1) < 0$. Rewriting as $\sqrt{2} \Erf^{-1}(2 \alpha - 1) \sqrt{\portfolio^\top \covmat \portfolio} = \frac{1}{2} \frac{2 \sqrt{2} \Erf^{-1}(2 \alpha - 1) }{ \sqrt{\portfolio^\top \covmat \portfolio} } \portfolio^\top \covmat \portfolio $, we can reinterpret the agent to have a risk aversion parameter of $2 \sqrt{2} \Erf^{-1}(2 \alpha - 1) / \sqrt{\portfolio^\top \covmat \portfolio}$ that simultaneously depends on the level $\alpha$ and also on the portfolio volatility $\sqrt{\portfolio^\top \covmat \portfolio}$. Thus, our setup here illustrates one of the core conflicts between the principal and the agent whereby they have strictly different risk preferences. Of course, without the constraint $v' \in \riskmeasureconstraints$, there will be no conflict between the principal and the agent.




%%%%%%%%%%%%%%%%%%%%%%%%%%%%%%%%%%%%%%%%%%%%%%%%%%%%%%%%%%%%%%%%%%%%%%%%%%
\section{Literature review}
%%%%%%%%%%%%%%%%%%%%%%%%%%%%%%%%%%%%%%%%%%%%%%%%%%%%%%%%%%%%%%%%%%%%%%%%%%
Primary references: \cite{patton2019dynamic}, \cite{chen2016semiparametric}

Secondary references: \cite{dimitriadis2019joint}

Portfolio choice: \cite{brandt1999estimating}

Mechanism design: \cite{rogerson1985first}, \cite{mirrlees1999theory} (written in 1975 but published in 1999), \cite{jewitt1988justifying}

Score models: \cite{creal2013generalized}, recent surveys \cite{artemova2022scoretheory, artemova2022scoreapplications}

Nonlinear regression: \cite{huber1967behavior}, \cite{white1984nonlinear}, \cite{oberhofer1982consistency}, \cite{powell1984least}, \cite{weiss1991estimating}

Quantile regression: \cite{koenker2005quantile, koenker2017quantile}

Financial econometrics: \cite{engle2004caviar}

Elicitable risks: \cite{fissler2016higher}, \cite{gneiting2011making},  recent survey \cite{he2022risk}



%%%%%%%%%%%%%%%%%%%%%%%%%%%%%%%%%%%%%%%%%%%%%%%%%%%%%%%%%%%%%%%%%%%%%%%%%%%
%% Bibliography 
%%%%%%%%%%%%%%%%%%%%%%%%%%%%%%%%%%%%%%%%%%%%%%%%%%%%%%%%%%%%%%%%%%%%%%%%%%%
\newpage
\bibliographystyle{apalike}
\bibliography{references}

\newpage


%%%%%%%%%%%%%%%%%%%%%%%%%%%%%%%%%%%%%%%%%%%%%%%%%%%%%%%%%%%%%%%%%%%%%%%%%%%%
%%% Appendix
%%%%%%%%%%%%%%%%%%%%%%%%%%%%%%%%%%%%%%%%%%%%%%%%%%%%%%%%%%%%%%%%%%%%%%%%%%%%
%\newpage
%\appendix
%\input{appendix.tex}




\end{document}
